%%=============================================================================
%% Methodologie
%%=============================================================================

\chapter{\IfLanguageName{dutch}{Methodologie}{Methodology}}%
\label{ch:methodologie}

Like any thesis, this one starts with a comprehensive literature study of the research domain and of the technologies used. The goal of this chapter is to get a clear answer to which of the different routes these packages could go on should be chosen. This literature study can be found in chapter~\ref{ch:stand-van-zaken}

After the literature study, the fake authentication packages will be created. The creation of these packages can be found in chapter~\ref{ch:fake-auth}. The first section explains the initial state of the application created by Mr. Vertonghen. The flow of data in the new fake authentication is further expanded upon. The design decisions are described and all the ways modular programming is used for separating classes and methods into independent pieces are shown.

The code is now written, and needs to be turned into the two packages and set free to the public so that it can be used in future projects. The creation publishing process of these packages is detailed and whether or not there were any problems encountered during this process.

Lastly a detailed way of how to install, implement and customize the packages is written down and eventually a medium.com article could be published, depending on whether or not this can be done in time. If this article is not able to be created, this explanation will of course still be available in the README of the packages.
