%%=============================================================================
%% Voorwoord
%%=============================================================================

\chapter*{\IfLanguageName{dutch}{Woord vooraf}{Preface}}%
\label{ch:voorwoord}

%% TODO:
%% Het voorwoord is het enige deel van de bachelorproef waar je vanuit je
%% eigen standpunt (``ik-vorm'') mag schrijven. Je kan hier bv. motiveren
%% waarom jij het onderwerp wil bespreken.
%% Vergeet ook niet te bedanken wie je geholpen/gesteund/... heeft

Before you lies the bachelors thesis `Fake authentication in Blazor: Creating a NuGet-package for faking authentication in a Blazor application for easier development`. This thesis is written to fulfil the graduation requirements of the  Applied Computer Science degree at HOGENT.

During the first semester of the school year '22-'23, I took the class 'Enterprise Web C\#' given by Mr. B. Vertonghen. In this class I was taught how to develop a complete application in the open-source web framework 'Blazor', developed by Microsoft. This also incorporated creating a fake authentication library which allowed for easy switching between regular users, admins, or users that aren't logged in.

During these classes Mr. Vertonghen also expressed his desire to simplify the process of authenticating a user, or at least temporarily doing so during the development stage of the app in question. This gave us the idea of creating a NuGet-package of the fake authentication for both front- and back-end.

I would like to thank my supervisor and cosupervisor, Mr. B. Vertonghen for the idea of this thesis and the guidance during the process of working on it. I learned a lot about creating a thesis, and also about myself.

I would also like to thank my family and friends for being there for me during tougher times.

I wish you much enjoyment in reading.

Jonathan Couck

Aalst, 13th of August 2023
